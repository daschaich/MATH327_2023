% ------------------------------------------------------------------
\documentclass[12 pt]{article} % A4 paper set by geometry package below
\pagenumbering{arabic}
\setlength{\parindent}{10 mm}
\setlength{\parskip}{12 pt}

% Nimbus Sans font should be reasonably legible
\usepackage{helvet}
\renewcommand{\familydefault}{\sfdefault}
\usepackage[T1]{fontenc}  % Without this \textsterling produces $

% Section header spacing
\usepackage{titlesec}
\titlespacing\section{0pt}{12pt plus 4pt minus 2pt}{0pt plus 2pt minus 2pt}
\titlespacing\subsection{0pt}{12pt plus 4pt minus 2pt}{0pt plus 2pt minus 2pt}
\titlespacing\subsubsection{0pt}{12pt plus 4pt minus 2pt}{0pt plus 2pt minus 2pt}

\usepackage{amsmath}
\usepackage{amssymb}
\usepackage{graphicx}
\usepackage{verbatim}    % For comment
\usepackage[paper=a4paper, marginparwidth=0 cm, marginparsep=0 cm, top=2.5 cm, bottom=2.5 cm, left=3 cm, right=3 cm, includemp]{geometry}
\usepackage[pdftex, pdfstartview={FitH}, pdfnewwindow=true, colorlinks=true, citecolor=blue, filecolor=blue, linkcolor=blue, urlcolor=blue, pdfpagemode=UseNone]{hyperref}

% Put module code and last-modified date in footer
\usepackage{fancyhdr}
\pagestyle{fancy}
\fancyhf{}
\renewcommand{\headrulewidth}{0pt}
\cfoot{{\small \thisunit}\hfill \thepage\hfill {\small \moddate}}

% Hopefully address Canvas complaints about pdf tagging
%\usepackage[tagged]{accessibility}
\hypersetup {
  pdfauthor={David Schaich},
  pdftitle={Statistical Physics Tutorial Activity},
}
% ------------------------------------------------------------------



% ------------------------------------------------------------------
% Shortcuts
\newcommand{\Om}{\ensuremath{\Omega} }
% ------------------------------------------------------------------



% ------------------------------------------------------------------
\begin{document}
\newcommand{\thisunit}{MATH327 Tutorial (Entropy)}
\newcommand{\moddate}{Last modified 1 Mar.~2023}
\begin{center}
  {\Large \textbf{MATH327: Statistical Physics, Spring 2023}} \\[12 pt]
  {\Large \textbf{Tutorial activity \ --- \ Entropy bounds}} \\[24 pt]
\end{center}

We met the second law of thermodynamics by considering what happens when two subsystems are brought into thermal contact --- allowed to exchange energy but not particles.
Conservation of energy means that if subsystem $\Om_1$ has energy $e_1$, the other subsystem $\Om_2$ must have energy $E - e_1$, where $E$ is the total energy of the overall micro-canonical system $\Om$.
We found (in Eq.~21 on page 33 of the lecture notes) that the total number of micro-states of the overall system is
\begin{equation*}
  M = \sum_{e_1} M_{e_1}^{(1)} M_{E - e_1}^{(2)}
\end{equation*}
where $M_e^{(S)}$ is the number of micro-states of subsystem $S \in \left\{1, 2\right\}$ with energy $e$.

Because $M$ is a sum of strictly positive terms, we can easily set bounds on it.
Say the sum over $e_1$ has $N_{\text{terms}} \geq 1$ terms $M_{e_1}^{(1)} M_{E - e_1}^{(2)}$, and define $\max$ be the largest of those terms.
Then $\max \leq M$, with equality only when $N_{\text{terms}} = 1$.
Similarly, $M \leq N_{\text{terms}} \cdot \max$, with equality when every term in the sum is the same.
All together, we have
\begin{equation*}
  \max \leq M \leq N_{\text{terms}} \cdot \max.
\end{equation*}

This can be more powerful than it may initially appear, thanks to the large numbers involved in statistical physics.
For illustration, suppose $\max \sim e^N$ and $N_{\text{terms}} \sim N$ for a system with $N$ degrees of freedom.
(We have already seen $M = 2^N = e^{N\log 2}$ for a system of $N$ spins with $H = 0$, while $H > 0$ introduces factors of $N!$ that \href{https://en.wikipedia.org/wiki/Stirling's_approximation}{Stirling's formula} can recast in terms of $N^N = e^{N\log N}$.)
Then
\begin{equation*}
  e^N \lesssim M \lesssim N e^N.
\end{equation*}
If we take the logarithm and recall $\log M = S$ is the entropy, this gives us
\begin{equation*}
  N \lesssim S \lesssim N + \log N.
\end{equation*}
With our characteristic $N \sim 10^{23}$, we have $\log N \sim 50$ and $10^{23} \lesssim S \lesssim 10^{23} + 50$, a very tight range in relative terms, with the upper bound only $\sim$$10^{-20}\%$ larger than the lower bound.

To see how this works in practice, let each of $\Om_1$ and $\Om_2$ be a spin system with $N_1 = N_2 = 10$ spins and $H = 1$.
Fix $E = -10$ for the combined system and numerically compute the bounds on its entropy,
\begin{equation*}
  \log\left(\max\right) \leq S \leq \log\left(N_{\text{terms}} \cdot \max\right).
\end{equation*}
What fraction of the true entropy $S$ is accounted for by $\log\left(\max\right)$?
How do these answers change for $N_1 = N_2 = 20$, $30$, $40$, $\cdots$, still with fixed $E = -10$?

By considering the sort of spin configurations that produce $\max$, you can see the emergence of an `arrow of time'!

\end{document}
% ------------------------------------------------------------------
