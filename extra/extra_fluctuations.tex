% ------------------------------------------------------------------
\documentclass[12 pt]{article} % A4 paper set by geometry package below
\pagenumbering{arabic}
\setlength{\parindent}{10 mm}
\setlength{\parskip}{12 pt}

% Nimbus Sans font should be reasonably legible
\usepackage{helvet}
\renewcommand{\familydefault}{\sfdefault}
\usepackage[T1]{fontenc}  % Without this \textsterling produces $

% Section header spacing
\usepackage{titlesec}
\titlespacing\section{0pt}{12pt plus 4pt minus 2pt}{0pt plus 2pt minus 2pt}
\titlespacing\subsection{0pt}{12pt plus 4pt minus 2pt}{0pt plus 2pt minus 2pt}
\titlespacing\subsubsection{0pt}{12pt plus 4pt minus 2pt}{0pt plus 2pt minus 2pt}

\usepackage{amsmath}
\usepackage{amssymb}
\usepackage{graphicx}
\usepackage{verbatim}    % For comment
\usepackage[paper=a4paper, marginparwidth=0 cm, marginparsep=0 cm, top=2.5 cm, bottom=2.5 cm, left=3 cm, right=3 cm, includemp]{geometry}
\usepackage[pdftex, pdfstartview={FitH}, pdfnewwindow=true, colorlinks=true, citecolor=blue, filecolor=blue, linkcolor=blue, urlcolor=blue, pdfpagemode=UseNone]{hyperref}

% Put module code and last-modified date in footer
\usepackage{fancyhdr}
\pagestyle{fancy}
\fancyhf{}
\renewcommand{\headrulewidth}{0pt}
\cfoot{{\small \thisunit}\hfill \thepage\hfill {\small \moddate}}

% Hopefully address Canvas complaints about pdf tagging
%\usepackage[tagged]{accessibility}
\hypersetup {
  pdfauthor={David Schaich},
  pdftitle={Statistical Physics Extra Practice},
}
% ------------------------------------------------------------------



% ------------------------------------------------------------------
% Shortcuts
\newcommand{\be}{\ensuremath{\beta} }
\newcommand{\vev}[1]{\ensuremath{\left\langle #1 \right\rangle} }
\newcommand{\pderiv}[2]{\ensuremath{\frac{\partial #1}{\partial #2}} }
% ------------------------------------------------------------------



% ------------------------------------------------------------------
\begin{document}
\newcommand{\thisunit}{MATH327 Extra (Fluctuations)}
\newcommand{\moddate}{Last modified 14 Apr.~2023}
\begin{center}
  {\Large \textbf{MATH327: Statistical Physics, Spring 2023}} \\[12 pt]
  {\Large \textbf{Extra practice \ --- \ Particle number fluctuations}} \\[24 pt]
\end{center}

Consider the fugacity expansion of the grand-canonical partition function,
\begin{equation*}
  Z_g(T, \mu) = \sum_{N = 0}^{\infty} \xi^N \, Z_N(T),
\end{equation*}
where the fugacity $\xi = e^{\be \mu} = e^{\mu / T}$ and $Z_N(T)$ is the $N$-particle canonical partition function (which is independent of $\xi$).
Recall that $\Phi(T, \mu) = -T \log Z_g(T, \mu)$ is the corresponding grand-canonical potential.

Derive a relation between the average particle number $\vev{N}$ and the derivative $\displaystyle \pderiv{}{\log \xi}\Phi = \xi \pderiv{}{\xi}\Phi$.

Derive a relation between $\vev{\left(N - \vev{N}\right)^2}$ and $\displaystyle \left(\xi \pderiv{}{\xi}\right)^2 \Phi$.

Specializing to Maxwell--Boltzmann statistics, for which the fugacity expansion simplifies to $Z_g^{\text{MB}}(T, \mu) = \exp[\xi Z_1(T)]$, show
\begin{equation*}
  \frac{\sqrt{\vev{\left(N - \vev{N}\right)^2}}}{\vev{N}} = \frac{1}{\sqrt{\vev{N}}}.
\end{equation*}

As an aside, this final result means that the relative fluctuations in the particle number vanish in the \textit{thermodynamic limit} $\vev{N} \to \infty$.
That is, when $\vev{N}$ is large it is approximately constant, which allows the grand-canonical system to be approximated by the corresponding canonical system with fixed $N$.

\end{document}
% ------------------------------------------------------------------
