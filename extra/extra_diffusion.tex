% ------------------------------------------------------------------
\documentclass[12 pt]{article} % A4 paper set by geometry package below
\pagenumbering{arabic}
\setlength{\parindent}{10 mm}
\setlength{\parskip}{12 pt}

% Nimbus Sans font should be reasonably legible
\usepackage{helvet}
\renewcommand{\familydefault}{\sfdefault}
\usepackage[T1]{fontenc}  % Without this \textsterling produces $

% Section header spacing
\usepackage{titlesec}
\titlespacing\section{0pt}{12pt plus 4pt minus 2pt}{0pt plus 2pt minus 2pt}
\titlespacing\subsection{0pt}{12pt plus 4pt minus 2pt}{0pt plus 2pt minus 2pt}
\titlespacing\subsubsection{0pt}{12pt plus 4pt minus 2pt}{0pt plus 2pt minus 2pt}

\usepackage{amsmath}
\usepackage{amssymb}
\usepackage{graphicx}
\usepackage{verbatim}    % For comment
\usepackage[paper=a4paper, marginparwidth=0 cm, marginparsep=0 cm, top=2.5 cm, bottom=2.5 cm, left=3 cm, right=3 cm, includemp]{geometry}
\usepackage[pdftex, pdfstartview={FitH}, pdfnewwindow=true, colorlinks=true, citecolor=blue, filecolor=blue, linkcolor=blue, urlcolor=blue, pdfpagemode=UseNone]{hyperref}

% Put module code and last-modified date in footer
\usepackage{fancyhdr}
\pagestyle{fancy}
\fancyhf{}
\renewcommand{\headrulewidth}{0pt}
\cfoot{{\small \thisunit}\hfill \thepage\hfill {\small \moddate}}

% Hopefully address Canvas complaints about pdf tagging
%\usepackage[tagged]{accessibility}
\hypersetup {
  pdfauthor={David Schaich},
  pdftitle={Statistical Physics Extra Practice},
}
% ------------------------------------------------------------------



% ------------------------------------------------------------------
% Shortcuts
\newcommand{\vdr}{\ensuremath{v_{\mathrm{dr}}} }
% ------------------------------------------------------------------



% ------------------------------------------------------------------
\begin{document}
\newcommand{\thisunit}{MATH327 Extra (Diffusion)}
\newcommand{\moddate}{Last modified 14 Apr.~2023}
\begin{center}
  {\Large \textbf{MATH327: Statistical Physics, Spring 2023}} \\[12 pt]
  {\Large \textbf{Extra practice \ --- \ Drift and diffusion}} \\[24 pt]
\end{center}

There has been an oil spill $1~\mathrm{km}$ away from a \href{https://en.wikipedia.org/wiki/Marine_protected_area}{marine protected area} (MPA).
$1000$ tonnes of oil are in the water.
You have been called on to estimate how much time is available for the government to take action to protect the MPA.
The motion of each oil droplet can be modeled as a one-dimensional random walk towards (or away from) the MPA, with diffusion constant $D = 2~\mathrm{m}/\sqrt{\mathrm{min}}$.
If the oil has a drift velocity $\vdr = 1~\mathrm{m}/\mathrm{min}$ towards the MPA, how long do we have before $10$~tonnes of oil is inside the MPA?
How much additional time will it take for the amount of oil inside the MPA to double to a total of $20$~tonnes?

If the oil were washing up on a shore instead of entering an MPA, the mathematics would be significantly more complicated, because each droplet's random walk would \textit{stop} once it reached the shore and left the water.
This is known as a \textit{first-passage process}.
Without attempting this more complicated calculation, explain whether it will take more time, less time or the same amount time for $10$ tonnes of the spilled oil to wash up on shore, compared to the MPA case considered above, with everything else the same.

\textbf{Hint:} \href{https://scipy.org}{SciPy} is one tool you can use to invert the error function
\begin{equation*}
  \mathrm{erf}(u) = \frac{1}{\sqrt{\pi}} \int_{-u}^u e^{-x^2} \; dx \equiv P
\end{equation*}
and find the $u \geq 0$ corresponding to a given $0 \leq P < 1$, as illustrated by the example code below.

\vfill
\begin{verbatim}
>>> import math
>>> from scipy import special
>>>
>>> sigmas = [0.682689492, 0.954499736, 0.997300204, \
...                        0.999936656, 0.999999427]
>>> for P in sigmas:
...   u = special.erfinv(P)
...   n = round(u * math.sqrt(2.0))
...   print("u = %.4f for P=%.7f (%d sigma)" % (u, P, n))

u = 0.7071 for P=0.6826895 (1 sigma)
u = 1.4142 for P=0.9544997 (2 sigma)
u = 2.1213 for P=0.9973002 (3 sigma)
u = 2.8284 for P=0.9999367 (4 sigma)
u = 3.5356 for P=0.9999994 (5 sigma)
\end{verbatim}

\end{document}
% ------------------------------------------------------------------
