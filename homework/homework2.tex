% ------------------------------------------------------------------
\documentclass[12 pt]{article} % A4 paper set by geometry package below
\pagenumbering{arabic}
\setlength{\parindent}{10 mm}
\setlength{\parskip}{12 pt}

% Nimbus Sans font should be reasonably legible
\usepackage{helvet}
\renewcommand{\familydefault}{\sfdefault}
\usepackage[T1]{fontenc}  % Without this \textsterling produces $

% Section header spacing
\usepackage{titlesec}
\titlespacing\section{0pt}{12pt plus 4pt minus 2pt}{0pt plus 2pt minus 2pt}
\titlespacing\subsection{0pt}{12pt plus 4pt minus 2pt}{0pt plus 2pt minus 2pt}
\titlespacing\subsubsection{0pt}{12pt plus 4pt minus 2pt}{0pt plus 2pt minus 2pt}

\usepackage{amsmath}
\usepackage{amssymb}
\usepackage{graphicx}
\usepackage{verbatim}    % For comment
\usepackage[shortlabels]{enumitem}
\usepackage[paper=a4paper, marginparwidth=0 cm, marginparsep=0 cm, top=2.5 cm, bottom=2.5 cm, left=3 cm, right=3 cm, includemp]{geometry}
\usepackage[pdftex, pdfstartview={FitH}, pdfnewwindow=true, colorlinks=true, citecolor=blue, filecolor=blue, linkcolor=blue, urlcolor=blue, pdfpagemode=UseNone]{hyperref}

% Put module code and last-modified date in footer
\usepackage{fancyhdr}
\pagestyle{fancy}
\fancyhf{}
\renewcommand{\headrulewidth}{0pt}
\cfoot{{\small \thisweek}\hfill \thepage\hfill {\small \moddate}}

% Hopefully address Canvas complaints about pdf tagging and title
%\usepackage[tagged]{accessibility}
\hypersetup {
  pdfauthor={David Schaich},
  pdftitle={Statistical Physics Homework},
}
% ------------------------------------------------------------------



% ------------------------------------------------------------------
% Shortcuts
\newcommand{\be}{\ensuremath{\beta} }
\newcommand{\la}{\ensuremath{\lambda} }
\newcommand{\Om}{\ensuremath{\Omega} }
\newcommand{\vev}[1]{\ensuremath{\left\langle #1 \right\rangle} }
\newcommand{\pderiv}[2]{\ensuremath{\frac{\partial #1}{\partial #2}} }
\newcommand{\showmarks}[1]{\rightline{\texttt{[#1 marks]}}} % \showmarks needs to follow a blank line!
% ------------------------------------------------------------------



% ------------------------------------------------------------------
\begin{document}
\newcommand{\thisweek}{MATH327 Homework 2}
\newcommand{\moddate}{Last modified 13 Apr.~2023}
\begin{center}
  {\Large \textbf{MATH327: Statistical Physics, Spring 2023}} \\[12 pt]
  {\Large \textbf{Homework assignment 2}} \\[24 pt]
\end{center}

\section*{Instructions}
Complete all four questions below and submit your solutions by file upload \href{https://liverpool.instructure.com/courses/60601/assignments/226601}{on Canvas}.
By submitting solutions to this assessment you affirm that you have read and understood the \href{https://www.liverpool.ac.uk/media/livacuk/tqsd/code-of-practice-on-assessment/appendix_L_cop_assess.pdf}{Academic Integrity Policy} detailed in Appendix L of the Code of Practice on Assessment and have successfully passed the Academic Integrity Tutorial and Quiz.
The marks achieved on this assessment remain provisional until they are ratified by the Board of Examiners in June 2023.
Clear and neat presentations of your workings and the logic behind them will contribute to your mark.
This assignment is \textbf{due by 17:00 on Wednesday, 3 May}.
Anonymous marking is turned on and I will aim to return feedback by 14 May. % Can use make-up lecture on the 15th to go over any generic issues...
% ------------------------------------------------------------------



% ------------------------------------------------------------------
\vfill
\section*{Question 1: Indistinguishable spins}
In Section~3.4.2 of the lecture notes, we computed the Helmholtz free energy for $N \gg 1$ indistinguishable spins (Eq.~45):
\begin{equation*}
  F_I(\be) = -NH - \frac{\log\left[1 - e^{-2(N + 1) \be H}\right]}{\be} + \frac{\log\left[1 - e^{-2 \be H}\right]}{\be}.
\end{equation*}
What are the corresponding expressions for the internal energy $\vev{E}_I$ and the entropy $S_I$?

\showmarks{6}

What are the first \textbf{two} non-zero terms in \textbf{each} low-temperature ($e^{-2\be H} \ll 1$) expansion of $\vev{E}_I$ and $S_I$?

\showmarks{12}

What is the first non-zero term in the high-temperature ($\be H \ll 1$) expansion of $\vev{E}_I$?
What are the first \textbf{two} non-zero terms in the high-temperature expansion of $S_I$?

\showmarks{12}
% ------------------------------------------------------------------



% ------------------------------------------------------------------
\newpage
\section*{Question 2: Mixed ideal gases}
Consider a mixture of two classical ideal gases in thermodynamic equilibrium, in a container of volume $V$ at temperature $T$, like that illustrated below.
Let $N_1$ and $N_2$ be the fixed particle numbers of the two gases.
Within each gas the particles are indistinguishable, but particles of one gas are distinguishable from particles of the other gas.
In particular, they have different masses $m_1$ and $m_2$, implying different thermal de~Broglie wavelengths $\la_1$ and $\la_2$.

\begin{center}\includegraphics[width=0.5\textwidth]{figs/mixed.pdf}\end{center}

\begin{enumerate}[label={(\alph*)}]
  \item Calculate the canonical partition function $Z$ and the Helmholtz free energy of the ($N_1 + N_2$)-particle mixture, approximating $\log(N_i!) \approx N_i\log N_i - N_i$.

  \showmarks{6}

  \item Calculate the internal energy $\vev{E}$ and the entropy $S$ of the mixture.
        What is the condition of constant entropy?

  \showmarks{6}

  \item Calculate the pressure $P$ of the mixture, and relate it to the pressures $P_1$ and $P_2$ of each gas in isolation (as illustrated below).

  \showmarks{6}
\end{enumerate}

\begin{center}\includegraphics[width=0.3\textwidth]{figs/red.pdf}\hspace{0.3\textwidth}\includegraphics[width=0.3\textwidth]{figs/blue.pdf}\end{center}
% ------------------------------------------------------------------



% ------------------------------------------------------------------
\newpage
\section*{Question 3: Thermodynamic cycle}
Consider the Diesel cycle defined by the $PV$~diagram shown below, in which the `compression' stage $1 \to 2$ and the `power' stage $3 \to 4$ are both adiabatic, while the pressure is constant during the `injection/ignition' stage $2 \to 3$, and the volume is constant during the `exhaust' stage $4 \to 1$.

\begin{center}\includegraphics[width=0.7\textwidth]{figs/Diesel.pdf}\end{center}

Calculate the efficiency of the Diesel cycle, $\eta_D$, in terms of the compression ratio $r \equiv V_1 / V_2 > 1$ and the cutoff ratio $C \equiv V_3 / V_2 > 1$, where $C < r$.

\showmarks{20}

Fixing the compression ratio $r$, compare $\eta_D$ to the efficiency of the Otto cycle.
Is the Diesel cycle more efficient than the Otto cycle, less efficient, or the same?
How does this comparison depend on the cutoff ratio $C$?

\showmarks{8}
% ------------------------------------------------------------------



% ------------------------------------------------------------------
\newpage % Doesn't all fit on same page as Diesel cycle question...
\section*{Question 4: Magnetization}
Consider a classical system of $N$ distinguishable, non-interacting `spins' in a lattice at temperature $T = 1 / \be$, where the value $s_n$ of each spin can vary \emph{continuously} in the range $-1 \leq s_n \leq 1$.
In an external magnetic field of strength $H > 0$, the internal energy of the system is $\displaystyle E = -H \sum_{n = 1}^N s_n$.

\begin{enumerate}[label={(\alph*)}]
  \item Calculate the canonical partition function $Z$ and the Helmholtz free energy $F$ of the system, both as functions of $\be H$.

        \textbf{Hint:} Just like the continuous momenta considered in Sections 4.1 and 8.1 of the lecture notes, you will need to integrate over the continuous $s_n$.

  \showmarks{8}

  \item The derivative of the Helmholtz free energy with respect to the magnetic field defines the magnetization
        \begin{equation*}
          \vev{m} = -\frac{1}{N} \pderiv{F}{H}.
        \end{equation*}
        Assuming finite $H > 0$, calculate $\vev{m}$ for this system as a function of $\be H$, and determine its low- and high-temperature limits, $\displaystyle \lim_{T \to 0} \vev{m}$ and $\displaystyle \lim_{T \to \infty} \vev{m}$.

  \showmarks{8}

\item Calculate the leading $T$-dependent correction to \textbf{each} of the low- and high-temperature limits of $\vev{m}$ from the previous part.

  \showmarks{8}
\end{enumerate}
% ------------------------------------------------------------------



% ------------------------------------------------------------------
\end{document}
% ------------------------------------------------------------------
